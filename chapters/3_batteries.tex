\chapter{Batteries}
\minitoc

This thesis deals with the use of electrochemical methods to study redox active materials for the storage of electrical energy in the form of batteries (with a short excursus on double-layer electrochemical capacitors at the very end). I batteries, the electrodes are not called working and auxiliary (aka counter) as for analytical electrochemistry, but rather positive and negative electrode (IUPAC convention).

Batteries as classified based on the reversibility of the redox reactions into primary and secondary battery, which means in a more intuitive way: non recharcable and rechargable.

\section{Non-rechargible battries}

Non-rechargable batteries are the first type of batteries to reach commercialization and low prices. A primary battery uses inexpensive material that spontanously react when the external circuit is close. This means the voltage of the cell is always positive. Upon time, the material of the electrodes are consumed during the electrochemical reaction until the reaction cannot anymore proceed. The electromotive force of the cell goes to zero as well as the voltage difference between the poles.

Non-rechargable batteries allowed for the diffusion of portable electronic device, from the TV remotes, wrist watches and the Walkman to earing implants. For such they need to deliver a voltage difference of a few Volt to power the circuit boards, enough "energy" to keep the device on for enough time to be useful. A third requirment for their diffusion is the low price, for being easily replaced in consumer electronics. 

\section{A brief hystorical overview}

If we take again the first example of battery made of zinc and copper, the voltage is quite low. 
Later metallic positive electrodes were sobstitute with oxides. 

\section{Performance of batteries}

The performance of energy storage system is accessed following two parameters: power and energy.
Power is the electric work that an energy storage system can produce while energy is the amount of charge that the cell can deliver. Here, I am speaking of energy storage systems beacause power and energy can be calculated for any (electro-) chemical system. It is in fact quite interesting to compare different technologies to find the most suited for a specific application. To compare technologies or different chemistries for the same type of storage systems, power and energy are computed per unit of mass or volume; we than speak of power density and energy density. As an example we can think of an electric power tool to need a high power output but not much energy is require, while a pace maker doesn't require much power but rather high energy. For some application the space is a constrain and there high energy density is of highest importance. For example, in transportation high energy density, both per unity of mass and volume) are a deciding factor.

Achieving high energy density or high power density (or both) can be technically challenging or economically expensive. To make another example, to store electrical energy from solar panel on the roof of public houses, energy density is not as important quite rather the absolute energy per unit of cost.

[ragone plot]

\section{Secondary batteries}

The redox reaction at the electrode is an equilibrium reaction that can always be reverted when applaying external energy from the outside. In common words we say charging. The sign of the current between the positive and negative terminal of the battery is positive (opposite), the electromotive force, enhce gibss free energy of reaction, is also positive (which means work is made from the environment to the system) and, ideally, the redox active material is brought back to its original state. This type of operation is called electrochemical cell while for spontaneous reaction (electromotive force negative) it is referred in electrochemistry as galvanic cell.

In real cases, reverting the reaction might be inefficient. In fact, many side product can be formed, such as metal oxides, metal hydroxides or gas evolution due to electrolyte (or electrode material) decomposition. Common materials used in primary batteries can actually be recharded but the process produces hydrogen from the split of water electrolyte and upon discharge cycling metal like zinc convert into oxide, passivating the surface of the negative electrode.

Efficiency is then the key to have a functioning secondary batteries. One can calulate the number of charge-discharge cycles that a battery can undergo based on the cycle efficiency. For commercial appliation the efficiency should be as close as possible to 100\%, from 99.95\% above.

The key for high efficiency is reversibility of the redox reactions. This is achieved using material that can intercalate ions; which means the ions are not converted to other chemical forms (metallic, oxide, ecc.) but rather stored as ions. Materials with ordered strucutres (ionic or covalent) are the candidate for this mechanisms. Pairing of them and ions can shuttle from one electrode to the other. Typical examples of such materials are transition metals oxide, phosphates, sulfates and graphite.

The redox active metal is part of the crystal, upon reduction, a cataion is hosted in the free spaces of the structure maitning electroneutrality. The opposite happen when inverting the current. The mechanisms is also known as "rocking-chair" because of the back and forth of the ions. Transition metals have also high reduction potential producing cells with high voltage difference and hence high power density.

This concept, paired with organic electrolytes, allowed to obtain lithium ion-batteries with astonishing efficiency that now power all our electronic devices and electric veichels.

\section{The choice of metallic ion for shuttling}

The lithium-ion battery is known by anyone nowadays, regardles the scientific background. Lithium is such a good candidate because is the smallest ion (in ionic radius) with single charge. This make it suitable for intercalating in many structures. In principle any metal can be used but the intercalation become more difficult increasing the radius and number of charges. The next logical candidate is sodium which indeed is seeing first commercialization in the latest few years (first li-ion battery based on organic elctrolyte was commercialized in XXX with a energy density of ?? while the first na-ion batteries was commercialized in XXXX with an energy density of XXX; at the same year li-ion energy density reached XXX). 

In princple any matal from hearth to trasnition can be used for such a purpose. As an example, zinc is already used in primary batteries for its abboundance. It all come to a matter of finding the right "host" for the ion would that be though intercalation or conversion. When the ionic radius is larger, though, it is more difficult to find a suitable crystal strucutre and when the charge per ion is two or three (Al), more then one electron transfer step is needed for the reduction. 

On the other hand, having a less reactive cation allows to use aqueos electrolyte which, despite a smaller operational voltage window, is cheaper and safer (not flammable, not toxic).

The research is exploring the use of potassium, calcium, zinc in both acqueos and organic electrolyte.

\section{Electrolytes}

Electrolytes are evaluated based on two properties: the voltage stability window and conductivity. Ideally one wants to obtain high values of both. The stability windows allows to use electrodes with a voltage difference high, while conductivity reduces the resistance to the ionic transport and the diffusivity of species in the diffuse layer of the interphase that affect the over-potential. Both characteristics ensure high power density. Note that the conductivity has a secondary effect beacause with high conductivity and small concentration over-potential, the same electrode can be used in its full capacity (suppose it has more plateau) inside the stability windows of the elctrolyte. Experimentally, the stabilty window is extimated sweeping the voltage from very low to very high values, the electrolyte is stable when the current that flows in the region is negligible. The most simple example is the one of water. The stability window of water is 1.2V vs SHE. In fact, from 1.2V the oxigen in water oxidizes to produce pure oxigen gas while below 0V vs SHE, hydrogen gas is evoleved. Of course this is a thermodynamic evaluation, kinetics of such reaction much be much lower than the redox at the electrode material. The window of aqeus solution can be extended working with the solute. In highly concentrated solution the windows can be kinetically extend increasing the reaction over-potential. In facto when water molecules are solavating an ion, they are blockd in the position by the electrostatic local field. In this case it is commonly used the name water in salt with electrolyte solution of 5 to 20 m. The extended voltage window comes with the price in conductivity beacause the high concentration reduces the mobility of the species. For battery application it is still a good trade-off abilitating the use of certain electrodes and increasing the final cell operating voltage.

Deep eutected point solvents?

The opposite case is the one of organic solvents that usually present wider stability window but lower conductivity. Organic solvents allow also to have in the electrochemical cell metals in the reduced form while the absense of oxigen ions to react with. 
The most used organic compound in batteries are cyclic carbonates which have wide voltage windows of 4 to 5 V but high viscosity at room temperature. To optain sufficient conductivity the viscosity is reduced by mixing them and balance their properties.

This compunds are unfortunalty flamable and toxic which present a problem for the saftey of the user as well as the environment in caso of leakage and during the disposal of exausted batteries. This is a topic of greate discussion in the community and for this reasons aques electrolytes (and electrodes for them) as well as solid-electrolyte (see later for more) are activelly reaserached with greate investments in the public and private sector.

\section{State of the art of Li-ion batteries}

\subsection{Positive electrodes}

To achieve high power density, the positive electrode should have an high electrochemical potential. Looking at the standard reduction potentials of the metals in the periodic tables is clear that thebest choice are transition metals. As mentioned before, it is conveninete for achieving high reversibility to shuttle a small cation inside a crystalline structure so the best candidates as redox active materials are oxides and  The state of the art for Li-ion batteries is NMC622 as positive electrode and graphite as negative electrode. 

\subsection{Graphite as negative electrode}

Graphite can reach 0V vs Li (-3 vs SHE) making it the lowest electrochemical potential from the periodic table. It is tempting  to think of sobstituting graphite with metallic lithium which has a nominal energy density XX time larger, but its drowback is the morfology of electrodeposition because lithium, especially at higher current desnities, doesn't electrodeposit uniformelly but rather form dendritic strucutres that easilly lead to short-circuit. 

[describe plateau of graphite]

\subsection{The solid-electrolyte interface}

Organic electrolytes start to reduce around 100mV vs Li forming various short chained organic compounds. At the same time, the potential is near to the formation of lithium flourides catalyzed by the realize of lfourine from the decomposition of the electrolyte as well as Lithium oxide and other inorganic compounds. All of these new phases tend to gem at the surface of the negative electrode where the electrochemical potential of the electrons is around 0VvsLi. A solid layer is then formed around the particles of the graphite (or lithium) at the negative electrode. This layer happen to be ionic conductive and doesn't impare the operation of the cell. This layer called solid-electrolyte interface forms as soon from the reduction of the negative electrode but stop growing soon. This creates a stable sitation where no more electrolyte is decomposed while the lithium ion can pass though it. The growth of the layer is important for stabilizing it early. A uniform layer will stop growing sooner, while a porous layer will continouing grow for many cycles of the cell. A uniform grows is catalyzed by slow current density on the negative electrode so after a battery cell is assembled, the first one or two cycled are carried out at vary small currents. This time-consuming step is referred to as "formation" and it is a crytical step of battery production. Because of its copmlexity it is difficult to control or predict. It is also difficult to evaluate exprimentally beacuse the layer is thin, soft and has the same composition of the rest of the cell. Many strategy has been proposed to get a uniform an short-growing SEI but the research is stil haighly active, but from the perspective of esperimetnal characterization and modelling. It is important to remark the importance of a SEI that stop growing soon. Despite being ionic conductive and not imparing the cell behaviour itself, its formation is a sign of electrolyte decomposition and reaction with lithium. When lithium reacts it is not capable anymore to be shuttled enche the total energy of the cell drops. This mechanisms of lithium loss is called "loss of lithium inventory" and it is one of the three may couses of battery degradation (also called "aging") together with the degration of elecrtoactive redox materials at the postive and negative electrodes.


\subsection{Solid electrolytes}

Once the li-ion batteries started to pleatue in energy and power density with the advent of NMC622, the research interest moved to high energy cathode which have the problem of structure instability and lack of stable organic electrolyte at voltage above 4.5V and on the other hand solid-state electrolytes to unlock the potentiality of metallic lithium.

Solid electrolytes are interesting becaouse the absense of liquid reduce the safety risk concerned to toxicity, gas evolution and lickage in the environment when damaged. A solid electrolyte can be also thinner than the usual polymeric separator that holds the liquid, increasing the power and energy density while keeping the same redox active materials. Furthermore, the hope of the scientific community is to abilitate the safe use of matellic negative electrodes (Li, Na, K,...) beacause a solid electrolyte  might suppress the grouwth of dendrides. Unfortunalty introduces other problems like the loss of surface contact between the metallic electrode and the solid electrolyte during reduction and oxidation as well as promoting the metal growth in the free spaces of the solid electrolytes. Nontheless it is one of the hottest topic of reserach at the moment from both universities and private copmanies. Latest results shows a ciclability of 500 cycles in pouch cell format. The scaling up of the production of such materials is also a matter of research.


\section{On the nomenclature}

In the battery technology field positive and negative electrode are usually referred to as cathode and anode. This names originate from erly time of electrochemistry. The cathode is where oxidation takes place, from greek, while anode is where the oxidation take place. For a primary battery that undergoes "sponatnous discharge", the positive electrode behave as a cathode while the negative electrode behaves like an anode. For secondary battery furing charge, i.e. operated as electrochenmical cell, the positive electrode undergoes reduction behaving as anode, while the negative electrode as cathode. Despite the change in current sign the positive electrode remains "postitive" in the sense of having an higher electrochemical potential compared to the other electrode of the cell.

It is anyway common, especially from scientist coming from the field of chemical synthesis and material science with a limited interst in electroanalytics, to call the electroedes cathode and anode. 

Another source of misconception is the equivalence in meaning of the words charge and capacity. The latter comes from the idea of maximum quantity of something that in case of batteries is the electronic charge. It has not to be confused with the term capacitance of the electrical elemetn capacitor. The therm capacity is widley used in the sector while the correct one would be total charge (measure in Coulombs or Ah in the battery world). When the charge is integrate in the time of operation of a battery, gives the electrical energy (or work) of the device.

In this text, the IUPAC convention will be followed.

\section{Manifacturing of batteries}

We have covered the chemical aspect of redox active materials and electrolyte but for producing commercial batteries there are many other practical aspects to consider to reduce the volume of the cell. 
The first thing to consider is the conductivity of the components. Ceramic materials are in fact electric insulator, so, to garantee an electrical contact with the external circuit, a conductive element is added usualy in the form of carbon particles. The redox active materials togethere with carbon is kept in place using polymer binders. The ration between the components is usually 90-5-5 but it is a design choice of based on the properties of the materials and the needs of the final results. The part of the composition which is not active material is a passive component that doesn't contribute with the reaction but increases weight and volume of the cell reducing the voltage and power density. For this reason the Ragone plot shown before is of such importance, beacuse includes in the full picture all the direct and indirect effect of the choice of the chemicals in the final performance of the device.

To make an electrode one other component is need: the current collector. It has the scope of trasnfering the elctron from the external circuit to the redox active material and also prived a rigid substrate on which adeher the particles of active material. In practicle terms the components state above (active material, binder and conductor) are first dispersed in a solvent capable of dissolving the polymeric chain, the so called "slurry" (which should be very homogenoesu, hard to achieve) is then spread on the foil of a current collector and the solvent evaporated. The drying process is crytical in industrial manifacturing beacuse the solvents used for NMC are toxic (PVDF) and the evapporation requires a lot of energy and time. Attempts to produce electrodes via estrution of the electroactive materials are unders reaasearch, but, when homogeneit is achieved, thei are not trivial to scale up for industrial production. 

Immediatly come the second cosideration which is the thicknes of the components. The particles of active material must be in close contact with the electrolyte to allow lithium to get in and out the crystal, so the electrode must be thin hilighy porous and not too thick. This limits the upper limit of the volume occupied of the active material while it is difficult to shring too mich the volume of the passive component. As state of the art, active materials and additive spread on the substrate have a thicknes of around 40 um while a porous polymer seprate the electrodes to avoid short circuit. The porosity of the separator and its morfology is important beacause the electroltyte is soaked in them and direct diffusion path must be garanteed. The thickness of separators can arrive down to 12 um. 

As current collectors alluminum and nickel are commonly imployed. Alluminum, which is cheaper than nickel cannot be used as negative electrode in lithium ion batteries because alluminium alloys with lithium.

Batteries can also be produced with different form factors, the most known one is the cilyndrical cell, but there is also the prismatic cell and the pouch cell. Each one with pros and cons.

\section{Battery degradation}

The maximum charge that a battery can store sets the operation time. Upon usage, the battery loses some of it because of side proceses. These losses regards both the amount of lithium and redox active material. Acive lithium ions can be lost during further SEI formation on late life time of the cell. Despite the SEI growth stops early in the life ot the cell it is still thermodynaically possible and if the morfology of the graphite particle changes, bacause of cracking for examples (discussed later) new SEI can form that include the reaction with lithium ion. Another possbility of losing active lithium is trhough electrodeposition, especially during charging at high current density (aka fast charging). The third plateau of graphite is a few mV above the thermodynamic reduction potential of lithium. Furthermore, the potential is not equivalent in all the point of the volume of the graphite, some might have higher or lower values, what is measured is an avarage. Then lithium can electrodeposit at the negative electrode. The electrodeposition, as alteady introduced, it is not a uniform processes and might lead to cylindrical shapes which can both: re-dissolved, if the critical radius is not reached, leading only to loss of efficiency of the discharge cycle or in the worst case detatch from the surface as a metal macroscopic particle and float in the electrlyte. Since no electrical contat is present, oxidation and dissolution back to the electrolyte might not happen and that ammount of lithium is loss from the total count.

The other majour source of degradetion for batteries is the loss of active material (both at positive and negative electrodes) for losing contact with some particles. This is espetialy true when at high current rates particles are subjected to a change in volume. Upon cycling the mechanical stress can induce cracks.

Other minor sources of degradation are the reaction of the metal current collector with the product of electrolyte degradation. Furthermore, alluminum from the positive electrode might slowly dissolve into solution and travel to the negative electro to alloy with lithium; reversing the alloying reaction is inefficient.

\section{Electrochemical characterization of batteries}

The characterization of the battery means knowing:

-specific power, 
-specific energy, 
-maximum charge.

This information is extracted experimentally from electrochemical tests based on constant current techniques. The idea behind the experiment is to charge and discharge the battery with constant currents and observe the degradation over time measure in loss, or fading, of charge. 

The choice of the current to perform the experiment is quite arbitrary and it is based on the concept of "charging-rate", usually called simply C-rate. The C-rate is the inverse of the time (in hours) needed to complitely charge the cell up to its maximum theoretical capacity from a state of complete discharge. Enche, the current to achieve it is then 
$$I=\textit{maxium charge } [Ah] * \textit{c-rate }[h^{-1}].$$ 
For lithium ion batteries, very slow charges are C/20 or C/10 used for example in the formation process. 1C is the stand for a lab test. A battery is considered at commercial standard when the maximum charge reaches 80\% of its original values after 1000 cycles. Fast charge is C/3 and goes to "extrimelly fast charge" at C/4 or C/5.

Because of the over-potential, cycling at 1C, will not allow to charge the battery up to the full capacity. A strategy to metigate it is to apply a constant voltage technique after the constant current, with a voltage value equal to the upper voltage limit of the constant current charge. This allow to intercalate some more ions in the strucutre withot goign outside the thermodynamic stability of the electrodes.
This appreach is common in battery cycling and it is referred to as constant-current-costant-voltage (CCCV) protocol.

[How much does it intercalate?]

\section{Fast charging}

For batteries in electrical mobility, especially cars, it became important in the last years the idea of charging them very fast, let's say a time comparable to the stop at the gas station to refill the oil tank of a car based on thermal engine. 20-10 minutes seems the goal for fast charging. It has to be noted that fast charging is always perfomed in a limited state of charge, usually between 20 and 80\%, in fact otherwise the over-potential would be too high and the lithium intercalation too ununiform to damage the particles of active material. Furthermore extremelly fast charge can be achieved only with silicon-graphite alloys with high concentration of silicon which bring in the problem of volume expansion. Furthermore at this power loads the termal dissipation is extrimelly important and require fine engineering of the whole battery pack and thermal management system.

Up to this time the record is hold by the private company Store Dot which acieve extremelly fast chargin from 20 to 80\% in 8 minuts with good termals. 