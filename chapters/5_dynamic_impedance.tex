\chapter{Dynamic Electrochemical Impedance spectroscopy: \\state of the art}
\minitoc

 [Da rifrasare] 
 Impedance is defined as the transfer function between  current input and voltage output of an electrical circuit that belong to the class of linear and time invariant system. As such the system must be stationary during the measurement, and the input-output relationship must be linear and causal, i.e. the complex-valued transfer function follows the Kramer-Kroning relationship. There are scenarios where the system is not stationary nor linear, for example because of inherent instability, while studying non-equilibrium reactions or while operating electrochemical devices for practical applications. In these setting it is still possible to obtain, both theoretically and experimentally, an "instantaneous" impedance. The idea was explored during the 70's from Creason, Smith, Bond and colleagues in a series of paper on the admittance of red-ox couples  under constant or sweeping perturbations \cite{creasonFourierTransformFaradaic1972}\cite{creasonFourierTransformFaradaic1972b}\cite{creasonFourierTransformFaradaic1973a}\cite{bondOnlineFFTFaradaic1977} using a multi-sine waveform as perturbation to excite specific frequencies in the electrochemical system and use the Short Time Fast Fourier Transform (STFFT) to compute amplitude and phase of voltage and current and obtain the impedance from their ratio. In the 90's, with the advent of faster computers, this methodology gained a new interest and development thanks to Popkirov, Darocicki, Házì and their colleagues \cite{popkirovNewImpedanceSpectrometer1992}\cite{popkirovValidationExperimentalData1993}\cite{darowickiFrequencyDispersionHarmonic1995}\cite{haziMicrocomputerbasedInstrumentationMultifrequency1997}\cite{schieweUnifiedApproachTrace1998}.
During the first years of this century, the method was refined from both experimental \cite{darowickiInfluenceAnalyzingWindow2002}\cite{sacciDynamicElectrochemicalImpedance2009}\cite{slepskiOnlineMeasurementCell2009} and theoretical \cite{darowickiInstantaneousImpedanceSpectra2000} \cite{darowickiTheoreticalDescriptionMeasuring2000} sides from Sacci, Harringon, Slepski and colleagues and the name dynamic electrochemical impedance spectroscopy (DEIS) started to take over.
In 2012 Breugelmans, Hubin and their colleagues extended the new contribution of Pintelon and Schoukens on the identification of non-stationary systems and the Best Linear Approximation to electrochemical systems, introducing an approach based again on the DFT of voltage and current signals followed by regression of the frequency spectra to extract, impedance, non-linearity, non-stationarities and noise; the method took the name of ORP-EIS or operando EIS \cite{breugelmansOddRandomPhase2012}. These concepts were recently covered in a comprehensive review paper\cite{hallemans2023electrochemical}. In parallel, also La Mantia and Battistel \cite{battistel2016analysis} introduced a method based on DFT and filtering for voltage and current signals to estimate the impedance at low frequency, in a range in which the standard STFFT method is biased by the drift. As multi-sine perturbation, several waveforms were used, such as pseudo-random sequences, ternary binary sequences, chirps\cite{darowickiDeterminationElectrodeImpedance2004}, sin function or pulses, as well as pseudo-log distributed multi-sine.