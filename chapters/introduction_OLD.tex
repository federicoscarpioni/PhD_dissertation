\chapter{Introduction}

Qui andrà l'introduzione "a imbuto".

\section{Overview}
Working on applied science topics requires an interdisciplinary approach. This is particolarly difficult beacause one needs to understand topics of different matters, differente languages and tools to find a new usuful outcome to benefit the science community or industry. This work tries to do excaly this, binding electrochemistry, signal process theory, mathematical optimization, system theory and some non-basic programming. 

\section{On the storage of energy}

...

This work deals with electrochemical energy storage devicea and in particular batteries. This will be the further discussed of next section.

\section{Electrochemical energy storage devices}
Electrochemical energy storage devices are:
\begin{itemize}
    \item Electrochemical capacitors
    \item Batteries
    \item 
\end{itemize}

Here I am not mentioning electrochemical energy \emph{production} devices which are also extrimelly important in the green energy trasnition and widly researched, although esulate the scope of the work here presented.

\section{The very basics of electrochemistry for battery science}

In this section, the basics of electrochemistry will be reviewed to give any reader a good start off for the rest of the book. It is not my intention to lean on the the details of the subject while a wide literature is easily available. At the end, a list of books and resources will be pointed for the specific topics for the interested reader.
The story of electrochemistry begins in the late 18th century when scientists first discovered the intimate connection between electricity and chemical reactions. In 1780, Italian scientist Luigi Galvani famously observed that a frog’s leg twitched when touched by metal probes, hinting at a link between biological tissues and electrical forces. Shortly after, Alessandro Volta, inspired by Galvani’s work, built the first electrochemical battery—the voltaic pile—in 1800. Volta’s stack of alternating metal discs separated by cloth soaked in saltwater produced a steady electric current, marking the birth of modern batteries. 

\subsection{Interdisciplinarity}

Electrochemistry is deeply intertwined with several scientific and engineering disciplines, influencing a wide range of fields. In materials science, it plays a key role in understanding how materials behave at the atomic and molecular levels, particularly in relation to conductivity, corrosion, and the development of advanced battery materials. Electrochemistry is essential for creating new electrode materials and electrolytes for batteries, fuel cells, and supercapacitors. It also intersects with physics, particularly in the study of energy transfer, electron flow, and ion behavior, with quantum mechanics and solid-state physics helping to explain how electrons move through electrochemical systems, such as advanced batteries and semiconductors. In chemical engineering, electrochemical principles are crucial for designing and optimizing processes like electroplating, corrosion protection, and the large-scale production of chemicals through electrolysis, such as hydrogen production from water. In environmental science, electrochemistry is vital for pollution control and sustainable energy solutions, including technologies like fuel cells for clean energy and electrochemical water treatment systems for purifying wastewater or capturing carbon. Electrochemistry also plays an essential role in biology and biochemistry, as many biological processes, including nerve impulses, muscle contractions, and cellular respiration, involve ion transfer and electrochemical gradients. The study of bioelectrochemistry is applied in developing biosensors and biofuel cells. In energy engineering, electrochemistry underpins renewable energy technologies like solar power, hydrogen energy, and grid energy storage, improving devices like batteries, electrolyzers, and fuel cells. Lastly, in nanotechnology, electrochemical processes are critical for developing nanomaterials and devices, where electrochemical techniques are often used to fabricate nanostructures, improve sensors, and create efficient catalysts. It is rare to find a researcher working in pure electrcrochemistry without any application in specific fields. Consequently a lot of confusion is created when entering the subject of electrochemistry from a different domain of science. 

The following of this section gives an overview of the main aspect of electrochesmitry important for the discussion on the results in chapter \ref{chap:applications}. It will be also highlighted the connection between battery science and electrochemistry as well as some specific terminology for indicating the same thing.

\subsection{The electrochemical cell}
Many types of electrochemical cells have been produced, with varying chemical processes and designs, including galvanic cells, electrolytic cells, fuel cells, flow cells and voltaic piles.

\subsection{Battery science}
I would like to conclude this intense section with my definition of battery science. It is a field in between material science, electrochemistry, electronic engineering and (after a certain scale) production engineering. One can arrive to develop, optimize or work with battery from one of the field and needs to acquire the knowledge of the others. This might create confusion on the meaning of certain words to describe the same concept. The rest of the paragraph will deal with clarifing these terminology.
\subsubsection{cell and battery}
\subsubsection{capacity and charge}
\subsubsection{cathode/anode, positive/negative and working/counter electrodes}


\section{The working principles of a battery}
A battery works by converting chemical energy into electrical energy through electrochemical reactions that occur between two electrodes—an anode (negative) and a cathode (positive)—separated by an electrolyte.  When a battery discharges, i.e. it behaves like a galvanic cells, energy is transferred from the battery to external. At the negative electrode the oxidation reaction takes place, electrons are generate and flows though the external load and the mass of the electrode increases, incorporating new ions. On the other side of the cell, the positive electrode, the reduction reaction takes place using the current coming from the external load and cations are released in the electrolytic solution. The electrolyte is a fundamental component as well as the electrodes, in fact it conduction the ions, sustaining the redox reaction while blocking the electron conduction. A redox reaction can in principle always reverse applying an external potential to the cell and flowing opposite current. This process is the charge of the battery and the cell beahves as an electrolytic cell. The polarity of the cell remains the same in the two processes while the direction of the reaction and hence the current flowing in the electrodes changes make the oxidation reaction at the positive electrode and reduction reaction at the negative electrode. Unfortunately not all the reaction are kinetically reversible or at least not with a sufficient Culombing efficiency, which means the ratio of cumulative charge during charge divided by the one during discharge. 

It is important to consider that different types of batteries exist. A specific class is called redox flow batteries. 
For the purpose of this work the word battery will only indicate devices that behaves like galvanic and electrolytic cells.

\section{Battery chemistries and their applications}
There are different atomistic mechanisms that produce a redox reaction and they can be used to fabricate a battery:
\begin{itemize}
    \item material conversion
    \item alloying
    \item intercalation/insertion
    \item 
\end{itemize}

%\section{Performance of a battery}

%\section{Definitions of states for a battery}

%\section{Electrochemical methods for battery characterization}

%\subsection{Galvanostatic cycling}

%\subsection{Potentiostatic}

%\subsection{CCCV protocol}

%\subsection{Electrochemical Impedance Spectroscopy}

%\subsection{Cell design}

%\subsection{Reference electrode placement}

%\section{Working principle of the potentiostat/galvanostat}

%\section{Aging of batteries}

%\section{System identification}

%\section{Signal theory}

%\section{Impedance spectroscopy from system identification}

%\section{optimization problems}

%\section{Non-stationary impedance for electrochemical systems}